\documentclass{article}
\usepackage{amsmath,amstext,amssymb}
\usepackage[ttscale=0.7]{libertine}
\usepackage[libertine]{newtxmath}
\usepackage{siunitx}
\usepackage{inconsolata}
\usepackage{wrapfig}

\usepackage{sectsty}
\allsectionsfont{\sffamily}

\usepackage{xcolor}
\usepackage[colorlinks=true,
            linkcolor=blue,
            urlcolor=blue,
            breaklinks,
            pdftex,
            allbordercolors=white]{hyperref}
\usepackage{graphicx}
\usepackage{booktabs}
\usepackage[version=4]{mhchem}
\usepackage{parskip}

\usepackage{listings}

\usepackage{datetime2}
\renewcommand{\DTMdisplaydate}[4]{#1-#2-#3}

\begin{document}

\lstset{language=C++,stringstyle=\ttfamily}

\begin{center}
    \Large{\textbf{\sffamily{Phase Identification Numbers in THAMES}}}
\end{center}
\begin{center}
    \large{Jeffrey W. Bullard}
\end{center}
\begin{center}
    \large{\DTMnow}
\end{center}

\vspace{0.25truein}
\tableofcontents

\vspace{0.25truein}
This document describes the different kinds of identification numbers in THAMES
and how the relate to one another.

\section{Introduction to ID numbers: ICs and DCs}
Every component in THAMES, whether it is an independent component (IC), a dependent
component (DC) or a phase, has \textit{two} ways to identify it:
\begin{itemize}
    \item its name (a C++ string)
    \item its identification number (an int)
\end{itemize}
The reason that each component needs an identification number is that properties of the
components (number of moles, atomic mass, \textit{etc}) are stored in different
arrays (or C++ vectors), and the elements of a vector are distinguished by their order
in the list; the first element is element 0, the second is 1, and so on.
So for example, suppose we have nine independent components (ICs) and we are
storing the number of moles of the independent components
in a vector called \verb!icmoles!. The vector might look like this:

\begin{verbatim}
icmoles[0] = 0.2 , icmoles[1] = 1.05, . . . , icmoles[8] = 0.47
\end{verbatim}

Now let's suppose that I want to change the number of moles of \ce{Si} in the system.
I need to know which element in \verb!icmoles! corresponds to \ce{Si}.  We do that
by assigning an integer to \ce{Si}, called an `id'.  For example, if I assign
the id number 2 to \ce{Si}, then I will always know that \verb!icmoles[2]! is the one
I need to change.

So every IC has a unique id number and a unique name.  Similarly, every DC has
a unique id number and a unique name.  Here is some basic information
about them:
\begin{itemize}
\item The IC and DC names are read from the GEMS DCH input file.
\item The order in which an IC name is read from the DCH files is equal to
    its id number.  For example, if \ce{Al} is the first IC name read from the
    DCH file, then the id number for \ce{Al} is 0.
\item The same is true for the DC names.  That is, the order in which a DC name
    is read from the DCH file is equal to its id number.
\item The IC names are stored in the \verb!ChemicalSystem! object,
    using a C++ vector called \verb!ICnames_!.  The DC names are also stored
    in the \verb!ChemicalSystem! object using a vector called \verb!DCnames_!.
\item The numbers of moles of each IC are stored in the \verb!ChemicalSystem!
    object using an array called \verb!ICmoles_!, and similarly the numbers
    of moles of each DC are stored using an array called \verb!DCmoles_!.
\item The total number of ICs is remembered in the \verb!ChemicalSystem! object
    with the integer called \verb!ICnum_!, and the total number of DCs is
    remembered with the integer called \verb!DCnum_!.
\end{itemize}

Notice that all of this information is stored in the \verb!ChemicalSystem! object.
In fact, the \verb!ChemicalSystem! object stores almost \textit{all} of the chemical
information in THAMES, and the other parts of THAMES gain access to that information
using functions that communicate with the \verb!ChemicalSystem! object.  Here is
some other date about the ICs and DCs that are stored there:
\begin{itemize}
    \item The molar masses of the ICs are stored in a vector called
        \verb!ICmolarmass_!, and the molar masses of the DCs are stored
        in a vector called \verb!DCmolarmass_!.
    \item For DCs, we need to know how many moles of each IC it contains in one
        formula unit.  This is stored in a two-dimensional vector (the same thing
        as a matrix) called \verb!DCstoich_!.  Each row of the matrix is a DC,
        and each column in that row is the number of units of an IC in that DC.
\end{itemize}

\paragraph{Example 1.}  Suppose I have a system with only three ICs in this order:
\ce{Al}, \ce{H}, and \ce{O}.  Furthermore, suppose that there are three DCs in this
system in this order: \ce{Al^{3+}}, \ce{Al(OH)_4^+}, and \ce{H2O}.  Based on this
information, we can write down what will be stored in all of the data structures
we have described so far.  See if you can understand each one.

\scriptsize{
\begin{tabular}{l p{4.0in}} \toprule
\textbf{Structure} & \textbf{Contents} \\ \midrule
\verb!ICname_! & \verb!ICname_[0] = "Al"! \hspace{0.1in} 
                 \verb!ICname_[1] = "H"!  \hspace{0.1in}
                 \verb!ICname_[2] = "O"! \\
                 & \\
\verb!DCname_! & \verb!DCname_[0] = "Al3+"! \hspace{0.1in}
                 \verb!DCname_[1] = "Al(OH)4+"! \hspace{0.1in}
                 \verb!DCname_[2] = "H2O@"! \\
                 & \\
\verb!ICmolarmass_! & \verb!ICmolarmass_[0] = 26.981! \hspace{0.1in}
                 \verb!ICmolarmass_[1] = 1.001! \hspace{0.1in}
                      \verb!ICmolarmass_[2] = 15.999! \\
                 & \\
\verb!DCmolarmass_! & \verb!DCmolarmass_[0] = 26.981! \hspace{0.1in}
                 \verb!DCmolarmass_[1] = 95.01! \hspace{0.1in}
                      \verb!DCmolarmass_[2] = 18.020! \\
                 & \\
\verb!DCstoich_! &     1 \hspace{0.25in} 0 \hspace{0.25in} 0 \newline
                       1 \hspace{0.25in} 4 \hspace{0.25in} 4 \newline
                       0 \hspace{0.25in} 2 \hspace{0.25in} 1 \\ \bottomrule
\end{tabular}
}

\normalsize{ }
\subsection{Accessing the data of an IC or DC}
C++ is designed in such a way that the stored data in an object are \textit{private},
which just means that they can only
be accessed directly from within the object itself.  If I want to use the
data from within the object I can look directly at the vector element that
holds it.  For example, if I want to assign the molar mass of the first
IC to a variable called \verb!mmass! in the \verb!ChemicalSystem.cc! file,
I can just write this:
\begin{lstlisting}
// Only works from within ChemicalSystem.cc
mmass = ICmolarmass_[0];
\end{lstlisting}

However, if I want to do the same thing from within a different object,
you have to communicate with \verb!ChemicalSystem! with a ``getter'' function:
\begin{lstlisting}
// Works in other files
mmass = chemsys_->getICmolarmass(0);
\end{lstlisting}
Notice that now the id number, 0, is an argument of the function, enclosed in
parentheses, instead of a vector index enclosed in square brackets.
THAMES has been written to include both getter and setter functions for nearly
all of the private data in each class, so you will use them most of the time

Next, suppose I want to change the number of moles of the third IC to the value
\num{3.14}.  To do that, I can type this:
\begin{lstlisting}
// Only works from within ChemicalSystem.cc
ICmoles_[2] = 3.14;
\end{lstlisting}

Alternatively, I can use the setter function,
\begin{lstlisting}
// Works in other files
chemsys_->setICmoles(2,3.14);
\end{lstlisting}

\subsection{Working with IC and DC id numbers}
Usually, one will not memorize the id number of a particular IC or DC,
but it is likely that you may know its name.  THAMES has functions that will
tell you the id number of a component with a given name.  For example,
the molar mass of the IC named \verb!Al! can be found this way:

\begin{lstlisting}
// Works in other files
int idAl = chemsys_->getICid("Al");
mmass = chemsys_->getICmolarmass(idAl);
\end{lstlisting}
Or you can even combine the two lines into one:
\begin{lstlisting}
// Works in other files
mmass = chemsys_->getICmolarmass(chemsys_->getICid("Al"));
\end{lstlisting}
In addition, there is also a function that will do this directly by name:
\begin{lstlisting}
// Works in other files
mmass = chemsys_->getICmolarmass("Al");
\end{lstlisting}

All three of these alternatives will accomplish the same task.  And there are 
similar functions for other IC and DC data as well.  (Almost) all the getter
functions start with \verb!get! and the setter functions start with \verb!set!,
and the next letter after that is always upper case.  You can scan through
all the function names and what they do in the \verb!ChemicalSystem.h! file.

\section{Phases}
\subsection{GEM phases work the same way \ldots}
The \verb!ChemicalSystem! object holds a master list of all the phases
that GEMS recognizes in a particular system (I will call these phases ``GEM phases'' to
distinguish them from microstructure phases below).  Again, those GEM phase names
are stored in the DCH file and they are read, in order, into a vector
called \verb!phasename_!.  The same object also holds the molar mass
in \verb!phasemolarmass_!, number of moles of each GEM phase in
\verb!phasemoles_!, and mass of each GEM phase in \verb!phasemass_!.
And you can access all of these data using the same kinds
of functions as for IC and DC components.  Here are just a few examples.

\begin{lstlisting}
// Get the phase name of the second phase
string pname2 = chemsys_->getPhasename(1);

// Get the molar mass of the fourth phase
double mm04 = chemsys_->getPhasemolarmass(3);

// Get the molar mass of the phase named Alite
double mmAlite = chemsys_->getPhasemolarmass("Alite");

// Set the moles of the sixth phase to 3.1
chemsys_->setPhasemoles(5,3.1);

// Get the mass of the phase named Gp
double gypsummass = chemsys_->getPhasemass("Gp");

\end{lstlisting}

In the same way that DCs can contain multiple ICs, so also GEM phases can contain
multiple DCs.  For example, the aqueous phase called ``aq-gen'' contains dozens
of DCs, one for each type of dissolved ion or molecule that can exist in the solution.
Another example is the GEM phase called ``Alite'', which has only one DC called ``C3S''.

There is a vector of vectors in the \verb!ChemicalSystem! object called
\verb!phaseDCmembers_! that stores a list of every DC that is associated with
each phase.  In addition, there are several functions that access the information
in \verb!phaseDCmembers_!:

\begin{itemize}
    \item \verb!chemsys_->getPhaseDCmembers(i)! will return a vector of all the
        DC id numbers that are associated with the GEM phase id number \verb!=i!.
    \item \verb!chemsys_->getPhaseDCmembers(pname)! will return a vector of all the
        DC id numbers that are associated with the GEM phase name \verb!pname!.
    \item \verb!chemsys_->getPhaseDCmembers(i,j)! will return the id number of the
        \verb!j!-th DC associated with the GEM phase id number \verb!i!.
\end{itemize}

\subsection{\ldots but phases are more complicated}
The microstructure aspect of THAMES makes the accounting of phases a little more
complicated because we map each microstructure phase to one or more
GEM phases.  For example, in THAMES we have a microstructure phase called \verb!C3S!
that is mapped to the GEM phase called \verb!Alite!.  And we usually have a
microstructure phase called \verb!GYPSUM! that is mapped to the GEM phase
called \verb!Gp!.

In fact, we could do all of the microstructure phases this way, so that every
microstructure phase maps to exactly one GEM phase.  Some aspects of THAMES would
be easier if we did that.  But in most cement systems there will be nearly 100
different phases, and some of them are very closely related.  For example,
GEMS defines at least seven different carbonated AFm type
phases that differ in small ways from each
other.\footnote{C$_4$AcH$_9$, C$_4$Ac$_{0.5}$H$_{10.5}$,
    C$_4$Ac$_{0.5}$H$_{12}$, C$_4$Ac$_{0.5}$H$_9$,
C$_4$AcH$_{11}$, C$_4$Fc$_{0.5}$H$_{10}$, and C$_4$FcH$_{12}$.}
But for representing the microstructure we
will find it convenient to combine
all seven of them together into a single microstructure phase called
\verb!AFMC!.  For this reason, we keep a separate list of microstructure phase
names and microstructure id numbers.  Both of these are defined by the user
in the \verb!chemistry.xml! file, and they are stored in the \verb!ChemicalSystem!
object with the vectors \verb!micphasename_! and \verb!micid_!.  As before,
there are getter and setter functions for these as well:

\begin{lstlisting}
// Get the name of the third microstructure phase in chemistry.xml
string pname = chemsys_->getPhasename(2);

// Get the microstructure id number of the ninth microstructure
// phase in chemistry.xml
int id8 = chemsys_->getMicid(8);

// Set the microstructure id number of the fourth microstructure
// phase to the value 7
chemsys_->setMicid(3,7);
\end{lstlisting}

THAMES has some functions to keep track of the mapping between ``GEM phases''
and ``microstructure phases'':

\begin{lstlisting}
// Get the list of all GEM phase id numbers
// belonging to the microstructure phase called AFMC
vector<int> gpids = chemsys_->getMicphasemembers("AFMC");

// Get the second GEM phase id that is associated with the
// microstructure phase called HEMIANH
int gpid = chemsys_->getMicphasemembers("HEMIANH",1);
\end{lstlisting}

One can also directly access all of the DCs that are associated with a given
microstructure phase with these functions:
\begin{itemize}
    \item \verb!chemsys_->getMicDCmembers(i)! will return the vector of all
        DC id numbers that are associated with the microstructure phase id
        number \verb!i!.
    \item \verb!chemsys_->getMicDCmembers(mname)! will return the vector of all
        DC id numbers that are associated with the microstructure phase name
        \verb!mname!.
    \item \verb!chemsys_->getMicDCmembers(i,j)! will return the id number of the
        \verb!j!-th DC that is associated with the microstructure phase id number
        \verb!i!.
\end{itemize}

\subsection{Microstructure phases grouped by their kinetic behavior}
In an earlier version of THAMES, the \verb!KineticModel! object kept
track of different lists of microstructure phases depending on whether
they are kinetically contolled, thermodynamically controlled, or readily
soluble.  However, this structure led to a lot of confusing mappings
between the different list indices and their associated microstructure id numbers.
Therefore, the \verb!KineticModel! just stores to which, if any, of
these catagories a given microstructure phase id \verb!i! belongs with
the boolean functions \verb!isKinetic(i)!, \verb!isThermo(i)!,
and \verb!isSoluble(i)!.
\begin{enumerate}
    \item \textbf{Kinetically controlled phases} are microstructure
        phases that change moles according to some kind of kinetic
        rate equation.  The list of the id numbers of these phases is still stored
        in the \verb!ChemicalSystem! object in a vector called \verb!kineticphase_!.
    \item \textbf{Thermodynamically controlled phases} are microstructure phases
        that have their moles determined by chemical equilibrium in the GEMS
        model. The list of
        the id numbers of these phases is stored in the \verb!ChemicalSystem!
        object in a vector called \verb!thermophase_!
    \item \textbf{Readily soluble phases} are microstructure phases that
        will dissolve completely during the first time step.  These are
        also stored in the \verb!ChemicalSystem! object as
        \verb!thermophase_! (\textbf{not} as solublephase) because their
        precipitation can be thermodynamically favorable at some point during
        the simulation.
\end{enumerate}

\paragraph{Example 2.}  Suppose I have a microstructure with five microstructure
phases given \textit{in this order} in the \verb!chemistry.xml! file:
\begin{itemize}
    \item Name = C3S, id = 2 (kinetically controlled)
    \item Name = C3A, id = 4 (kinetically controlled)
    \item Name = Calcite, id = 5 (thermodynamically controlled)
    \item Name = K2SO4, id = 7 (readily soluble)
    \item Name = Ca(OH)2, id = 8 (thermodynamically controlled)
\end{itemize}
Here is how these data will be stored in the \verb!ChemicalSystem! object.
See if you can understand each one.

\scriptsize{
\begin{tabular}{l p{4.0in}} \toprule
\textbf{Structure} & \textbf{Contents} \\ \midrule
\verb!kineticphase_! & \verb!kineticphase_[0] = 2! \hspace{0.1in} 
                 \verb!kineticphase_[1] = 4! \\
                 & \\
\verb!thermophase_! & \verb!thermophase_[0] = 5! \hspace{0.1in}
\verb!thermophase_[1] = 7! \hspace{0.1in}
                 \verb!thermophase_[2] = 8! \\ \bottomrule
\end{tabular}
}

\normalsize{ }

\section{Exercises}

\begin{enumerate}
    \item Write a line or two of C++ code that will determine the atomic mass of
        an IC named ``C''.
    \item Write a line or two of C++ code that will determine the atomic mass of
        a DC named ``SiO2''.
    \item Write a line or two of C++ code that will determine how many moles of
        oxygen are contained in every mole of the DC named ``Al2O3''.
    \item Write a line or two of C++ code to change the mass of a microstructure phase called
        ``SFUME''.
    \item (Challenging) Write some C++ code that will determine how many moles
        of an IC named ``S'' are associated with GEM phase called ``ettr30''.
    \item (Very challenging) Write some C++ code to increase the number of moles of an IC
        called ``Mg'' due to the dissolution of 0.5 moles of a microstructure phase
        called ``MYPHASE'' that is associated with a GEM phase called ``Mphs"
\end{enumerate}

\end{document}
