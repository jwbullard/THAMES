\documentclass{article}
\usepackage{amsmath,amstext,amssymb}
\usepackage[ttscale=0.6]{libertine}
\usepackage[libertine]{newtxmath}
\usepackage{siunitx}
\usepackage{inconsolata}
\usepackage{wrapfig}

\usepackage{sectsty}
\allsectionsfont{\sffamily}

\usepackage{xcolor}
\usepackage[colorlinks=true,
            linkcolor=blue,
            urlcolor=blue,
            breaklinks,
            pdftex,
            allbordercolors=white]{hyperref}
\usepackage{graphicx}
\usepackage{booktabs}
\usepackage[version=4]{mhchem}
\usepackage{parskip}
\usepackage{fancyvrb}
\usepackage{fvextra}
\DefineShortVerb{\!}

\usepackage{datetime2}
\renewcommand{\DTMdisplaydate}[4]{#1-#2-#3}

\begin{document}

\begin{center}
    \Large{\textbf{\sffamily{Setting Up THAMES v2.5 Input}}}
\end{center}
\begin{center}
    \large{Jeffrey W. Bullard\footnote{Special thanks to Dr. Dmitrii Kulik who
    provided invaluable guidance for building GEMS3K and GEM-Selektor.}}
\end{center}
\begin{center}
    \large{\DTMnow}
\end{center}

\vspace{0.25truein}
\tableofcontents

\vspace{0.25truein}
This document provides guidance on how to build THAMES on various operating systems
The build process from scratch is somewhat complicated due to the need
to build and install the GEMS3K standalone library and the GEM-Selektor software.
Separate sections are given for Mac OS, Linux and Windows.

\section{\label{sec:mac}Mac OS}
\subsection{Software Requirements}
The following software must be installed and running properly on your computer
to build and install all the components:
\begin{itemize}
  \item Git software versioning system.
\item Cmake version 3.0 or later.
\item A C++11 compiler.
  This document will assume the Gnu C/C++ compiler suite with full support for
  C++11.
\item Doxygen version 1.18.13 or later. This is needed only for
  generating the documentation.
\item (Optional) \LaTeX typesetting software.
  This is needed only for building the PDF versions of the documentation.  A
  recent installation of \TeX Live will suffice.
\end{itemize}
This document will assume that all of these packages have been installed already.

\subsection{Downloading the Software}
Create a working directory somewhere in your home path.  For this document, the working
directory will be called !$WORKDIR!.  You should substitute the path to your working
directory everywhere you see that.

The remainder of these instructions require that you enter commands on the command line
(the Terminal app or iTerm2).  These commands will be typeset in monospace
font to distinguish them from other instructions.

\subsubsection{THAMES 2.5}
If you already have THAMES installed from github and you want to keep any local changes you
have made, then go to the directory where
you installed it and execute these commands:
\begin{enumerate}
    \item !git stash!
    \item !git pull!
\end{enumerate}
The first command stashes away your local changes so they won't be lost.  The second
command pulls all the updates from the remote repository.  Later, if you want to
put your local changes back into the updated version, you can run the command
!git stash pop!

On the other hand, if you have never installed THAMES on your computer, then
follow these steps:
\begin{enumerate}
        \item !cd $WORKDIR!
        \item !git clone https://github.tamu.edu/jwbullard/THAMES.git!
\end{enumerate}
Please contact the developer if you discover that GitHub prevents you
from cloning the repository. You may need to be added as a collaborator.

\subsubsection{GEM-Selektor}
GEM-Selektor is only needed to create basic thermodynamic input files for
THAMES. The downloadable binaries for GEM-Selektor can be found
at \url{https://gems.web.psi.ch/GEMS3/downloads/index.html}.

The installer is simple to use.  Double-clicking on it will open a window
from which you can just drag the gems3k application to where
you want it to reside.  For convenience, it is helpful to put it somewhere
within your own home folder and to make sure that the folder is also in your `PATH`.
For these instructions, we will call this folder !PathToGems!.
You should substitute the actual path to this folder on your computer in these instructions.

\subsubsection{\label{sec:thirdpartymac}Install GEM-Selektor Third-Party Thermodynamic Databases}
You will need \textit{at a minimum} the Cemdata18 database, which is already a
part of the THAMES git repository.

Next, ensure that GEM-Selektor is not running. Navigate to the GEM-Selektor DB.default
folder using either the command line or the Finder:

From the command line:

\begin{Verbatim}[breaklines=true]
cp $WORKDIR/THAMES/src/Cemdata18.01/* PathToGEMS/gems3.app/Contents/Resources/DB.default/.
\end{Verbatim}

Using the Finder:
\begin{enumerate}
  \item Navigate to the folder !PathToGems!
  \item Right-click on !gems3! icon, and select ``Show Package
    Contents''.
  \item Continue navigating within the package contents to
    !Contents/Resources/DB.default!
  \item In a separate Finder window, navigate into to the folder
    !WORKDIR/THAMES/src/Cemdata18.01/! and drag all the
    contents of that folder into the !DB.default! folder
    of the gems3 app from step 3 above.
\end{enumerate}

\subsection{Build and Install GEMS3K Standalone Library}
These commands will install the library:
\begin{enumerate}
        \item !cd $WORKDIR/THAMES/src/GEMS3K-standalone!
        \item !./install.sh!
\end{enumerate}

\subsection{Test GEM-Selektor}
You should now be able to run GEM-Selektor by double-clicking on the ``GEMSelektor''
icon wherever you installed it.

\subsection{Build and Install THAMES}
If all the previous steps have been executed successfully, installing THAMES should be pretty
straightforward.
\begin{enumerate}
    \item !cd $WORKDIR/THAMES/build!
    \item !cmake ..!
    \item !make!
    \item !make install!
\end{enumerate}
The last step will put the thames and vcctl2thames executables into the directory

!$WORKDIR/THAMES/bin!

\section{\label{sec:linux}Linux}
Installation and use of THAMES has not been thoroughly tested on Linux
computers. Please alert the developer of any changes to the instructions
that you discovered to enable successful installation.

\subsection{Software Requirements}
The following software must be installed and running properly on your computer
to build and install all the components:
\begin{itemize}
  \item Git software
  versioning system.
\item Cmake version 3.0 or later.
\item A C++11 compiler.
  This document will assume the Gnu C/C++ compiler suite with full support for
  C++11.
\item Doxygen version 1.18.13 or later. This is needed only for
  generating the documentation.
\item (Optional) \LaTeX typesetting software.
  This is needed only for building the PDF versions of the documentation.  A
  recent installation of \TeX Live will suffice.
\item The packages
  build-essentials, libx11-xcb-dev, and libglu1-mesa-dev. These can be installed
  on Linux using a command like !sudo apt-get install build-essentials!.
\end{itemize}
This document will assume that all of these packages have been installed already.

\subsection{Downloading the Software}
Create a working directory somewhere in your home path.  For this document, the working
directory will be called !$WORKDIR!.  You should substitute the path to your working
directory everywhere you see that.

The remainder of these instructions require that you enter commands on the command line.
These commands will be typeset in monospace font to distinguish them from other instructions.

\subsubsection{THAMES 2.5}
If you already have THAMES installed from github and you want to keep any local changes you
have made, then go to the directory where
you installed it and execute these commands:
\begin{enumerate}
    \item !git stash!
    \item !git pull!
\end{enumerate}
The first command stashes away your local changes so they won't be lost.  The second
command pulls all the updates from the remote repository.  Later, if you want to
put your local changes back into the updated version, you can run the command
!git stash pop!

On the other hand, if you have never installed THAMES on your computer, then
follow these steps:
\begin{enumerate}
        \item !cd $WORKDIR!
        \item !git clone https://github.tamu.edu/jwbullard/THAMES.git!
\end{enumerate}
Please contact the developer if you discover that GitHub prevents you
from cloning the repository. You may need to be added as a collaborator.

\subsubsection{GEM-Selektor}
GEM-Selektor is only needed to create basic thermodynamic input files for
THAMES.
The downloadable binaries for GEM-Selektor can be found
at \url{https://gems.web.psi.ch/GEMS3/downloads/index.html}. Follow the
installation instructions.

\subsubsection{\label{sec:thirdpartylinux}Install Third-Party Data Repositories}
You will need \textit{at a minimum} the Cemdata18 database, which is already a
part of the THAMES git repository.

Next, ensure that GEM-Selektor is not running. Navigate to the GEM-Selektor DB.default
folder using the command line:

\begin{Verbatim}[breaklines=true]
cp $WORKDIR/THAMES/src/Cemdata18.01/* PathToGEMS/gems3.app/Contents/Resources/DB.default/.
\end{Verbatim}

\subsection{Build and Install GEMS3K Standalone Library}
These commands will install the library:
\begin{enumerate}
        \item !cd $WORKDIR/THAMES/src/GEMS3K-standalone!
        \item !sudo ./install.sh!
\end{enumerate}
If you do not have administrator privileges, then you may need to work with your IT
help desk to get this installed.

\subsection{Build and Install THAMES}
If all the previous steps have been executed successfully, installing THAMES should be pretty
straightforward.
\begin{enumerate}
    \item !cd $WORKDIR/THAMES/build!
    \item !cmake ..!
    \item !make!
    \item !make install!
\end{enumerate}
The last step will put the thames and vcctl2thames executables into the directory
!$WORKDIR/THAMES/bin!

\section{\label{sec:windows}Windows}
\subsection{MSYS installation}
\begin{enumerate}
  \item Direct link to the installer is \url{https://github.com/msys2/msys2-installer/releases/download/2022-06-03/msys2-x86_64-20220603.exe}
  \item Use default installation folder
  \item Start MSYS2 MinGW x64 from the Start menu, then execute the command
    !pacman -Syu!
  \item The MSYS window will close automatically after that command.
  \item Start MSYS2 MSYS from the Start menu, then execute the following commands:
  \begin{enumerate}
    \item !pacman -Syu!
    \item !pacman -S --needed base-devel mingw-w64-x86_64-toolchain!
    \item !pacman -S mingw-w64-x86_64-cmake!
    \item !pacman -S mingw-w64-x86_64-make!
    \item !pacman -S git!
  \end{enumerate}
  \item Close the MSYS window
\end{enumerate}
 
\subsection{Download THAMES source code}
Start MSYS2 MinGW x64 from the Start menu and execute the command

!git clone https://github.com/jwbullard/THAMES.git!

You will need a username and password for GitHub.  Contact the developer
if you are unable to to clone the repository once you have established
your GitHub credentials.  You may need to be added to the project as a 
collaborator.
 
\subsection{GEM-Selektor Installation}
GEM-Selektor is only needed to create basic thermodynamic input files for
THAMES.
The downloadable binaries for GEM-Selektor can be found
at \url{https://gems.web.psi.ch/GEMS3/downloads/index.html}. Follow the
installation instructions.

\subsubsection{\label{sec:thirdpartywindows}Install Third-Party Data Repositories}
You will need \textit{at a minimum} the Cemdata18 database, which is already a
part of the THAMES git repository.

Next, ensure that GEM-Selektor is not running.
In one Windows File Explorer window, navigate to the GEM-Selektor DB.default
folder. It should be found at

!C:\Users\"your name"\GEMS39\Gems3-app\Resources\DB.default!

In another Windows File explorer window, navigate to

!WORKDIR/THAMES/src/Cemdata18.01!

Select all the files of this folder and drag them to the DB.default folder
in the other File Explorer window above.

\subsection{Build GEMS3K-Standalone}
Start MSYS MinGW x64 from the Start menu if it is not already open. Within the
MSYS window execute the following commands:
\begin{enumerate}
  \item !cd THAMES/src/GEMS3K-Standalone!
  \item !mkdir build!
  \item !cd build!
  \item
    \begin{Verbatim}[breaklines=true]
      cmake .. -G "MinGW Makefiles" -DCMAKE_CXX_FLAGS=-fPIC -DCMAKE_BUILD_TYPE=Release -DCMAKE_INSTALL_PREFIX=../Resources
    \end{Verbatim}
  \item !/mingw64/bin/mingw32-make.exe!
  \item !/mingw64/bin/mingw32-make.exe install!
\end{enumerate}
 
\subsection{Build THAMES}
\begin{enumerate}
\item !cd ../../../build!
\item
  \begin{Verbatim}[breaklines=true]
    cmake .. -G "MinGW Makefiles" -DCMAKE_BUILD_TYPE=Release -DCMAKE_INSTALL_PREFIX=../bin
  \end{Verbatim}
\item !/mingw64/bin/mingw32-make.exe!
\item !/mingw64/bin/mingw32-make.exe install!
\end{enumerate}
 
\subsection{Edit Windows Path}
\begin{enumerate}
  \item Start Edit the System Variables form the Start menu
  \item Click ``Environment Variables'' button
  \item Under ``System Variables'', scroll down until you see Path
  \item Select that line, and press ``Edit''
  \item Edit Environment variable window will come up
  \item Press ``New''
  \item Add !C:\msys64\mingw64\bin! to the bottom
  \item OK $\longrightarrow$ OK $\longrightarrow$ OK
\end{enumerate}
 
At this point, the THAMES executable, !thames.exe!,
should be under

\begin{Verbatim}
  C:\msys64\home\<username>\THAMES\bin
\end{Verbatim}
 
\end{document}
