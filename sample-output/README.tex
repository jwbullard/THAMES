\documentclass{article}
\usepackage{amsmath,amstext}
%\usepackage[ttscale=0.85]{libertine}
\usepackage{sectsty}{}
%\usepackage[libertine]{newtxmath}
\usepackage{booktabs}
\usepackage{siunitx}
\usepackage{parskip}
\usepackage{listings}
\usepackage{color}
\definecolor{gray}{rgb}{0.4,0.4,0.4}
\definecolor{darkblue}{rgb}{0.0,0.0,0.6}
\definecolor{cyan}{rgb}{0.0,0.6,0.6}

\lstdefinelanguage{XML}
{
  morestring=[b]",
  morestring=[s]{>}{<},
  morecomment=[s]{<?}{?>},
  stringstyle=\color{black},
  identifierstyle=\color{darkblue},
  keywordstyle=\color{cyan},
  morekeywords={xmlns,version}% list your attributes here
}

\lstset{
  language=XML,
  basicstyle=\ttfamily,
  columns=fullflexible,
  showstringspaces=false,
  commentstyle=\color{gray}\upshape
}

\usepackage[version=4]{mhchem}
\allsectionsfont{\normalfont\sffamily\bfseries}

\bibliographystyle{pnas}

\begin{document}
\begin{center}
\Large{\textbf{\textsf{User Guide for THAMES Input}}}
\end{center}
\begin{center}
\today
\end{center}

\vspace{0.25truein}
\section{Introduction}
THAMES is a program for simulating 3D microstructure development during cement
binder hydration, leaching, or sulfate attack.  It uses equilibrium
thermodynamic calculations to determine the types and amounts of various solid
phases that are in chemical equilibrium with a pore solution.  Briefly, this
requires a partitioning of the system into that which is kinetically controlled
and that which is thermodynamically controlled.  The kinetically controlled part
is usually composed of the cement clinker phases or any others that gradually
dissolve during cement hydration.  The thermodynamically controlled part is
usually the pore solution and any solid hydration products that can potentially
precipitate during the process.  More detailed information on the system
description can be found elsewhere~\cite{Bullard11a}.

The user is expected to know how to work with GEMS-Selektor to generate the
various thermodynamic system data files that are required for the GEM3K library
that is linked to THAMES.  More information on the contents of the thermodynamic data
data files and how to generate them may be found in the online documentation for
GEM-Selektor~\cite{Kulik13}.

\section{Launching THAMES}
The current version of THAMES is launched from the command line with a command
similar to this, assuming that your current working directory contains the
executable:
\lstset{language=bash}
\begin{lstlisting}
nohup ./thames < input.txt >& output.txt &
\end{lstlisting}
where \texttt{input.txt} is the top-level input data file and \texttt{output.out} is a file
to which standard output will be written.

\section{The Top-Level Input File}
The top-level input file is really just a list of other input file names that
THAMES will read at the beginning of its execution.  An example input file is
given below, followed by a line-by-line description.

\begin{lstlisting}[frame=trBL]
1
solution-dat.lst
solution-dbr.dat
gem3k-dat.lst
gem3k-dbr.dat
job1-ccr168-chemistry.xml
ccr168_45.img.thames
phasemod.txt
parameters.xml
myjob
\end{lstlisting}

\begin{itemize}
	\item The first line is a numerical code that tells THAMES which kind of a
	simulation is to be performed.  Choices are:
		\begin{itemize}
			\item 1 = Quit program
			\item 2 = Hydration
			\item 3 = Leaching with neutral water
			\item 4 = External sulfate attack
		\end{itemize}
	\item The next two lines are the names of the thermodynamic data files for just the pore
	solution.  The upper bound metastibility contraints for all DC solid phases
	must be set to a very small number such as \num{1e-30}, while the
	constraints for all DC solute components are set to a large number such as
	\num{1e6}.  Furthermore, all the DC names must be unique in this file.
	Otherwise, these two files are identical to the data files
	described on the fourth and fifth lines.  Therefore, usually one will
	generate the whole-system files from GEM-Selektor and then make copies
	to edit for the solution data files.
	\item The fourth and fifth lines are the names of the thermodynamic data files for the
	entire system, including the solid hydration products that may precipitate.
	The DC names need not be unique in this file, but one may still wish to
	set the upper bound metastability constraint to be exceedingly low for
	some solid phases, such as quartz, that are known to not precipitate during
	cement hydration.
	\item The next line is the name of the XML-formatted ``THAMES phase definition file'' that defines
	the behavior of
	each phase, such as whether it is part of the kinetic partition or the
	thermodynamic partition, the degree of randomness of its growth shape, and
	the correspondence between THAMES-defined phases in the microstructure and
	the GEM-defined phases in the thermodynamic data files.  This file is
	somewhat complicated and will be described more fully in Section~\ref{sec:phasedefinitions}.
	\item The next line contains the name of the 3D initial microstructure file.  After a short
	header section that defines the size of the system, the main body of this
	file is a sequence of rows with one whole number on each row, with the whole
	number being the integer identification number of the THAMES phase at a
	particular location in the microstructure.  This file will be described more
	fully in Section~\ref{sec:microstructure}.
	\item The eighth line is the name of a file that contains the Young's
	modulus, in \si{\giga\pascal}, and Poisson ratio of each phase in the
	microstructure.  It is used only for computing effective elastic moduli
	of the system or for computing the volumetric expansion under sulfate
	attack.  An example line from the file might be
	\begin{lstlisting}
	2 C3S 117.6 0.314
	\end{lstlisting}
	where the first entry is the microstructure identification number for the
	phase recognized by THAMES and the second entry is the THAMES name of the phase.
	Both of these items are given in the THAMES phase definition file.  The
	third and fourth entries are the Young's modulus (\si{\giga\pascal}) and
	Poisson's ratio, respectively.
	\item The ninth line is the name of the XML-formatted file that defines
	the individual times, in days, at which the system's state is to be evaluated
	and updated.  After a short header defining the XML file and schema
	location, a typical line in the file might be

	\lstset{language=XML}
	\begin{lstlisting}
	<calctime>0.0174494</calctime>
	\end{lstlisting}
	At the very end of the file is another line that determines the time
	intervals at which the full 3D microstructure is to written to a file.
	For example,
	\begin{lstlisting}
	 <image_frequency>1.0</image_frequency>
	\end{lstlisting}
	means that a microstructure is to be written for each day of simulation
	time.
	\item The last line is the root name of all output files that will be
	created by the simulation.  It can be any string you wish.
	\end{itemize}

\section{\label{sec:phasedefinitions} THAMES Phase Definition File}
The THAMES phase definition file is an XML file, the form of which
is validated against the schema definition file defined at the end of
the second line of the file:
\begin{lstlisting}
<chemistry_data xmlns:xs="http://www.w3.org/2001/XMLSchema-instance"
xs:noNamespaceSchemaLocation="/Users/username/thames_wd/chemistry.xsd">
\end{lstlisting}
This file has two basic sections.  The first section containes several entries
that pertain to the entire simulation, and the second section specifies
the definition and behavior of each phase that THAMES recognizes.
\subsection{System-wide parameters}
The first six lines are the system-wide parameters:
\begin{lstlisting}
  <numentries>22</numentries>
  <blaine>408.0</blaine>
  <refblaine>385.0</refblaine>
  <wcRatio>0.45</wcRatio>
  <temperature>298.15</temperature>
  <reftemperature>296.15</reftemperature>
 \end{lstlisting}
 where \texttt{numentries} is the number of microstructure phases recognized by
 THAMES---and also the number of subsequent \texttt{phase} code blocks in the
 remainder of this file.  The entry \texttt{blaine} is the Blaine fineness of
 the cement powder (\si{\meter\squared\per\kilo\gram}), which is used by the
 kinetic dissolution model for the portland cement clinker components.  The
 \texttt{refblaine} entry is used to calibrate the kinetic model and
 should not be changed by the user.  The \texttt{wcRatio} parameter is the water-cement
 mass ratio of the binder being simulated.  The \texttt{temperature} entry is
 the system temperature (\si{\kelvin}), and \texttt{reftemperature} is a
 reference temperature used by the GEMS library and by the THAMES kinetic model.
 It should not be changed by the user.

 \subsection{THAMES phase definitions}
 As described earlier, THAMES defines phases that are to be kinetically
 controlled by some internal kientic model of dissolution or growth, and also
 phases that are controlled by thermodynamic equilibrium.  The change in amount
 of a kinetically controlled phase during a particular time interval is
 determined by a kinetic rate equation such as the model by Parrot and
 Killoh~\cite{Parrot84,Lothenbach06}.  In contrast, the change in amount
 of a thermodynamically controlled phase during that same interval the amount
 required to reach chemical equilibrium with the pore solution.

 \subsection{Kinetically controlled phases}
 We document the meaning of each line of a kinetically controlled phase input
 block using XML comments here.  An XML listing is first provided, followed
 by the explanation of each line.
 \begin{lstlisting}
   <phase>
    <thamesname>C3S</thamesname>
    <id>2</id>
    <porosity>0.0</porosity>
    <gemphase_data>
      <gemphasename>Alite</gemphasename>
      <gemdcname>C3S</gemdcname>
    </gemphase_data>
    <interface_data>
      <randomgrowth>0.0</randomgrowth>
      <growthtemplate>2</growthtemplate>
      <growthtemplate>12</growthtemplate>
      <growthtemplate>19</growthtemplate>
      <growthtemplate>20</growthtemplate>
      <affinity>
        <affinityphaseid>2</affinityphaseid>
        <affinityvalue>1</affinityvalue>
      </affinity>
    </interface_data>
    <impurity_data>
      <k2ocoeff>0.00087</k2ocoeff>
      <na2ocoeff>0.0</na2ocoeff>
      <mgocoeff>0.00861</mgocoeff>
      <so3coeff>0.007942</so3coeff>
    </impurity_data>
    <kinetic_data>
      <type>kinetic</type>
      <scaledmass>54.46</scaledmass>
      <k1>1.5</k1>
      <k2>0.05</k2>
      <k3>1.1</k3>
      <n1>0.7</n1>
      <n3>3.3</n3>
      <Ea>41570.0</Ea>
      <critdoh>2.0</critdoh>
    </kinetic_data>
    <display_data>
      <red>162.0</red>
      <green>117.0</green>
      <blue>95.0</blue>
      <gray>220.0</gray>
    </display_data>
  </phase>
  \end{lstlisting}

Each phase definition is begun and ended by the \verb!<phase>! \ldots
\verb!<\phase>! identifiers.
\begin{itemize}
	\item \texttt{thamesname} is the ASCII character string that THAMES uses
	to identify a particular microstructure phase.
	\item \texttt{id} is the whole number identifier for the phase in the
	microstructure.  These are the numbers used in the microstructure input
	file.
	\item \texttt{porosity} is the \textit{internal} porosity associated with
	the phase.  This includes and porosity that is too small to be resolved at
	the scale of the model.  \texttt{C3S} has no internal porosity and so its
	value is 0 in this example.  But other phases like C--S--H have gel porosity
	that could be included in this value.  The porosity is the volume fraction
	of the phase at the bulk scale which is porous
\end{itemize}

\subsubsection{GEM phase data}
This block sets up the correspondence between the THAMES phase in the
microstructure and the phase name(s) and dependent component name(s) linked to that
phase in the GEM chemical system definition, which can be found in the DBR data
file.  In the current example, the GEM phase name is \texttt{Alite} and the
dependent component is \texttt{C3S}.  However, users may wish to associate
multiple GEM phases and dependent components with a single THAMES phase.  As an
example, the user may wish to define a single microstructure phase for
AFt, so the THAMES name could be \texttt{AFt} and have a unique microstructure
identifier.  However, the GEM chemical system definition includes multiple
phases and dependent components that can be associated with AFt.  The block
below shows how to include all the types known to the GEM chemical system
definition (DBR).

\begin{lstlisting}
    <gemphase_data>
      <gemphasename>ettringite-Al</gemphasename>
      <gemdcname>ettringite</gemdcname>
      <gemdcname>Fe-ettringite</gemdcname>
    </gemphase_data>
    <gemphase_data>
      <gemphasename>ettringite-Fe</gemphasename>
      <gemdcname>1ettringite</gemdcname>
      <gemdcname>1Fe-ettringite</gemdcname>
    </gemphase_data>
    <gemphase_data>
      <gemphasename>SO4_CO3_AFt</gemphasename>
      <gemdcname>tricarboalu</gemdcname>
      <gemdcname>2ettringite</gemdcname>
    </gemphase_data>
    <gemphase_data>
      <gemphasename>CO3_SO4_AFt</gemphasename>
      <gemdcname>1tricarboalu</gemdcname>
      <gemdcname>3ettringite</gemdcname>
    </gemphase_data>
    <gemphase_data>
      <gemphasename>ettringite</gemphasename>
      <gemdcname>4ettringite</gemdcname>
    </gemphase_data>
\end{lstlisting}

In this case there are five GEM phases that all can be identified as AFt. 
Moreover, each of these phases is modeled as a solid solution with two
end-members that are each dependent components, so each GEM phase has two
DC entries.  Check the DBR file to ensure that the phase names and DC names
are typed exactly as they appear in the DBR file.

\subsubsection{Interface data}
This block is identified by the \verb!<interfacedata>! \ldots
\verb!<\interfacedata>! identifiers.  The block defines roughly
the growth morphology and the locations where the phase can grow.
\begin{itemize}
	\item \texttt{randomgrowth} is a number on $\left[ 0,1 \right]$ that determines
	how much randomness there is to the
	growth process of a phase.  When a phase is eligible to grow, THAMES creates
	a list of all the sites in the microstructure where growth of that phase is
	allowed, and then orders the list in descending order of the energetic
	favorability of that site for growth.  The energetic favorability is
	determined by the ``affinity'' of the growing phase for each of the phases
	in its neighboring sites.  Suppose that $n$ voxels of the phase need to
	grow during a given iteration.  If \texttt{randomgrowth} = 0, then the first
	voxel will be placed at the location of the first member of the growth list
	(energetically most favorable), the second will be placed in the second
	member, and so on until all $n$ voxels have been placed.  However,
	if the \texttt{randomgrowth} parameter is $r > 0$, then a fraction $r$
	of the list members will be randomly shuffled in pairs to create a
	disordered list.  The growing voxels are then assigned from the top of the
	list as before.
	\item \texttt{growthtemplate} is a THAMES phase identification number. 
	There may be multiple growth templates defined for any given phase. When
	constructing the list of sites at which a given phase can grow, THAMES will
	add an empty site to that list if one or more of the neighboring sites is
	occupied by a phase that belongs to the list of growth templates.  One
	should always define the phase itself as one of its own growth templates.
	\item \texttt{affinityphaseid} and \texttt{affinityphasevalue} provide a
	way to specify with more detail the local energetic favorability of
	a potential growth site.  The \texttt{affinityphasevalue} is positive
	if a growing phase has an affinity for growing next to the phase with
	the corresponding \texttt{affinityphaseid}, and is negative if the two
	phases are not energetically disposed to grow next to each other.  The
	notion of ``affinity'' might be imagined as a rough proxy for wettability according
	to the Young-Dupr{\'{e}} equation, so that the affinity will be negative
	if the contact angle between the two phases is greater than \SI{90}{\degree}
	and is positive if the contact angle is less than \SI{90}{\degree}.
	
	A phase with negative affinity for a growing phase should not be
	a member of the growing phase's growth templates.  And among all the
	growth templates, the default affinity is zero so that one needs to
	identify only those phases in the list of growth templates that have
	a positive affinity.
\end{itemize}

\subsubsection{Impurity data}
Each phase in cement clinker generally contains impurities that dissolve into
the pore solution as the phase itself dissolves.  The kinetic model in THAMES
for clinker phase dissolution currently recognizes four impurities, defined
on an oxide basis, that can
exist in each clinker phase, namely \ce{K2O}, \ce{Na2O}, \ce{MgO}, and
\ce{SO3}.  The impurity levels are defined on a mass fraction basis as
described by Lothenbach and Winnefeld~\cite{Lothenbach06}.  The data
are entered in the XML block identified by \verb!<impurity_data>! \ldots
\verb!<\impurity_data>!.

\subsubsection{Kinetic data}
The block identified by \verb!<kinetic_data>! \ldots
\verb!<\kinetic_data>! provides parameter values for the
kinetic model of clinker phase dissolution described by
Parrot and Killoh~\cite{Parrot84}.  The parameters in this block
are
\begin{itemize}
	\item The \texttt{type} parameter determines how a given phase in the
	original system, before any hydration, is treated by the model.  There are
	three options:
	\begin{itemize}
		\item \textbf{kinetic} phases will dissolve at a rate determined by
		the internal kinetic model, which is currently based on that of
	Parrot and Killoh~\cite{Parrot84}.  The number of moles of each
	independent component (IC) dissolved from these phases during a time step
	is transferred to the thermodynamic subsystem.
		\item \textbf{soluble} phases dissolves completely during the first
		time interval of the simulation.  All the IC moles for these phases
		are immediately transferred to the thermodynamic subsystem in the first time step.
		\item \textbf{thermo} phases are those phases that may be present in
		the initial microstructure but which nevertheless should be considered
		part of the thermodynamic partition.  These phases are not dissolved
		but their IC moles are still transferred to the thermodynamic
		subsystem.
	\end{itemize}
	\item \texttt{scaledmass} is the mass of the phase per \SI{100}{\gram}
	of solids in the original microstructure.  If the microstructure was
	created using VCCTL software, then this value should correspond to the
	mass fraction of that phase that was specified to create the microstructure.
	\item \texttt{k1}, \texttt{k2}, \texttt{k3}, \texttt{n1}, \texttt{n3}, and
	\texttt{Ea}, and \texttt{critdoh} are empirical parameters determined for
	each phase in the Parrot and Killoh model.  Refer to Lothenbach and
	Winnefeld~\cite{Lothenbach06} or to Parrot and Killoh~\cite{Parrot84} for
	more details.
\end{itemize}

\subsubsection{Display data}
The data in this section are used when creating a visualization of the
microstructure.  Each phase can be assigned a color defined by the
8-bit \texttt{red}, \texttt{green}, and \texttt{blue} channels (\num{0}
to \num{255}).  In addition, one can define an 8-bit monochromatic \texttt{gray}
value that will be used to create phase contrast when simulating a
backscattered electron (BSE) image of the microstructure.  For that purpose,
bright phases such as ferrite and alite should have relatively high
gray values while porosity should have low gray values.

\subsection{Thermodynamically controlled phases}
Thermodynamically controlled phases consist of the pore solution and any solids
that precipitate during hydration.  These phases have the same types of data as kinetically
controlled phases with two exceptions: they have neither a
\verb!<kinetic_data>! \ldots \verb!<\kinetic_data>! block
nor a \verb!<impurity_data>! \ldots \verb!<\impurity_data>! block.

\section{\label{sec:microstructure} THAMES Microstructure File}
Microstructure files for THAMES input have a header section as shown
in the following example:

\begin{lstlisting}
Version: 7.0
X_Size: 100
Y_Size: 100
Z_Size: 100
Image_Resolution: 1.00
\end{lstlisting}
The \texttt{Version} parameter is not used.  \verb!X_Size!, \verb!Y_Size!, and
\verb!Z_Size! are the number of voxels (or lattice sites) in the $x$, $y$, and
$z$ axes of the microstructure.  \verb!Image_Resolution! is the size of a single
voxel in \si{\micro\meter} units.

The remainder of the microstructure file is a sequence of
\verb!X_Size!$\times$\verb!Y_Size!$\times$\verb!Z_Size! lines, each of which has
the integer id of the THAMES phase occupying
the voxel associated with that line.  The lines are ordered so that the $x$
coordinate varies most quickly, followed by the $y$ coordinate and lastly by
the $z$ coordinate.  As an example of the coordinate ordering for a $3 \times 3 \times 2$
microstructure is
\begin{align*}
(0,0,0) \\
(1,0,0) \\
(2,0,0) \\
(0,1,0) \\
(1,1,0) \\
(2,1,0) \\
(0,2,0) \\
(1,2,0) \\
(2,2,0) \\
(0,0,1) \\
(1,0,1) \\
(2,0,1) \\
(0,1,1) \\
(1,1,1) \\
(2,1,1) \\
(0,2,1) \\
(1,2,1) \\
(2,2,1)
\end{align*}

\subsection{Transforming a VCCTL microstructure}
In the majority of cases, an initial THAMES microstructure will be based on a 3D
microstructure created within VCCTL.  Unfortunately, a VCCTL microstructure file
cannot be read directly by THAMES, but instead must be transformed so that the
phase identifiers agree.  \texttt{vcctl2thames} is a short program that can perform the
transformation for you.

\bibliography{Myrefs}
\end{document}